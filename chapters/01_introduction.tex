\chapter{Introduction}


\section{Background and Context}
\label{section:background}

Laparoscopic surgery offers a minimally invasive option, significantly cutting down on patients' postoperative pain and recovery time. Yet, the intricate dexterity this technique demands has created a clear need for robotic assistance. As a result of this need systems like Intuitive Surgical's da Vinci have been operating since 1999 \cite{Lanfranco2004RoboticSurgery}. While these systems have proven highly effective at reducing recovery time, they also come with prohibitive price tags and technical complexities, which makes them largely out of reach for many researchers. These barriers hinder further advancements in robotic assisted surgery and as a result, researchers at MCI previously developed a low cost, desktop-operated surgical robotic system.

\section{Problem Statement}
\label{section:problem_statement}

While this system was effective in providing a low cost alternative to existing solutions, its dynamic performance and software architecture left room for improvement. The software architecture was implemented using a C-based Arduino framework, which lacked many features necessary for complex robotic system development. Additionally, the system was originally controlled with a tuned PID controller. This controller, while providing basic control over the system, lacked the control necessary to carry out more complex motion paths. The system was also highly nonlinear in nature, and as the control target deviated from the system origin point, its performance degraded considerably, resulting in inadequate control behavior.

\section{Research Aim and Objectives}
\label{section:objectives}

The goal of the research conducted in this thesis was twofold. First, to implement a Robotic Operating System 2 (ROS 2) framework for the existing system. ROS 2 is the current industry standard open-source distributed robotics middleware framework, making it well suited for the dynamic, precision-focused requirements of surgical robotic systems. Its modular nature allows for the development of multiple complex subsystems in parallel, and its framework can be easily adapted to accommodate future additions or improvements to the platform. Additionally, ROS 2 offers many essential tools for complex robotic systems, such as system visualization, collision detection, and frame transformation frameworks. Porting the existing system to this ROS 2-based framework would allow access to these benefits and facilitate further development of the system in the future.

Secondly, due to the lack of dynamic performance and control in the current system, this research aimed to develop a more robust characterization and control strategy to address its nonlinear nature. The improved characterization strategy involves comparing the results of step response experiments to closed-loop excitation data to better model the system. More advanced control strategies, such as Linear Quadratic Regulator (LQR) and Linear Quadratic Integral (LQI) controllers with feedforward control, could then be applied to this system model. Additionally, gain scheduling techniques would be utilized to adapt the controller to the system’s highly nonlinear behavior.


\section{Scope}
\label{section:scope_limitations}

This thesis focuses on the implementation of an adaptive control framework within a ROS 2 based architecture for a teleoperative laparoscopic surgical robotic test platform. The scope of the research includes the migration of the existing robotic system from a C-based Arduino framework to ROS 2, the development of advanced control strategies to address system nonlinearities, and the experimental validation of the proposed control methods using a laboratory-scale test platform.

Clinical application and in vivo testing are beyond the scope of this work; thus, the system is evaluated exclusively in controlled lab environments. The research prioritizes software architecture and control performance over hardware design improvements. Furthermore, the adaptive control methods are tested on predefined motion trajectories and system responses, which may not cover the full range of operational scenarios encountered in actual surgical procedures.


