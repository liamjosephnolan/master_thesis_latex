\chapter{Introduction}


\section{Background and Context}
\label{section:background}

Laparoscopic surgery offers a minimally invasive surgical option that significantly reduces postoperative pain and recovery time for patients. However, the complex dexterity demands of this surgical technique have created a need for robotic assistance. Systems such as Intuitive Surgical’s da Vinci have been performing operations since 1999. While highly effective, these systems are prohibitively expensive and technically complex, and as a result, remain largely inaccessible to many researchers, surgeons, and clinicians. These barriers make it difficult to advance developments in the field of robotic-assisted surgery. In an effort to reduce these obstacles, researchers at MCI have previously developed a low-cost, desktop-operated surgical robotic system.

\section{Problem Statement}
\label{section:problem_statement}

While this system was effective in providing a low-cost alternative to existing solutions, its dynamic performance and software architecture left room for improvement. The software architecture was implemented using a C-based Arduino framework, which lacked many features necessary for supporting further technical advancements. Additionally, the system was controlled using a tuned PID controller which, while providing basic control over the system, left significant performance gains unrealized. The system was also highly nonlinear in nature, and as the control target deviated from the linearization point, its performance degraded considerably, resulting in inadequate control behavior.

\section{Research Aim and Objectives}
\label{section:objectives}

The goal of the research conducted in this thesis was twofold. First, to implement a ROS 2 framework for the existing system. ROS 2 is the current industry standard open-source distributed robotics middleware framework, making it well suited for the dynamic, precision-focused requirements of surgical robotic systems. Its modular nature allows for the development of multiple complex subsystems in parallel, and its framework can be easily adapted to accommodate future additions or improvements to the platform. Additionally, ROS 2 offers many essential tools for complex robotic systems, such as system visualization, collision detection, and frame transformation frameworks. Porting the existing system to this ROS 2-based framework would allow access to these benefits and facilitate further development of the system in the future.

Secondly, due to the lack of dynamic performance and control in the current system, this research aimed to develop a more robust characterization and control strategy to address its nonlinear nature. The improved characterization strategy involves comparing the results of step response experiments to closed-loop excitation data to better model the system. More advanced control strategies, such as Linear Quadratic Regulator (LQR) and Linear Quadratic Integral (LQI) controllers with feedforward control, could then be applied to this system model. Additionally, gain scheduling techniques would be utilized to adapt the controller to the system’s highly nonlinear behavior.


\section{Scope and Limitations}
\label{section:scope_limitations}

This thesis focuses on the implementation of an adaptive control framework within a ROS 2-based architecture for a laparoscopic surgical robotic test platform. The scope of the research includes the migration of the existing robotic system from a C-based Arduino framework to ROS 2, the development of advanced control strategies to address system nonlinearities, and the experimental validation of the proposed control methods using a laboratory-scale test platform.

Clinical application and in vivo testing are beyond the scope of this work; thus, the system is evaluated exclusively in controlled lab environments. The research prioritizes software architecture and control performance over hardware design improvements. Furthermore, the adaptive control methods are tested on predefined motion trajectories and system responses, which may not cover the full range of operational scenarios encountered in actual surgical procedures.

Limitations of the study include the use of a prototype platform that may not fully replicate the mechanical complexities of commercial surgical robots. The ROS 2 implementation is limited to core middleware functionalities, and some advanced features such as real-time communication optimizations and security enhancements are not fully explored. Additionally, due to resource constraints, long-term reliability and robustness tests were not conducted. These limitations highlight potential areas for future research to extend the applicability and performance of the system.

\section{Thesis Structure}
\label{section:thesis_structure}

This thesis is organized into seven chapters. Chapter 1 introduces the research background, problem statement, and research objectives. Chapter 2 reviews the current state of the art in laparoscopic surgical robotics and control methods. Chapter 3 outlines the system design and methodology used to develop the ROS 2-based framework and adaptive control strategies. Chapter 4 details the implementation process, including software architecture and integration challenges. Chapter 5 presents the experimental results and performance evaluation of the developed system. Chapter 6 discusses the findings, limitations, and implications of the research. Finally, Chapter 7 concludes the thesis and proposes directions for future work.
