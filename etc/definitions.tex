% deutsche Anpassungen
%\usepackage[ansinew]{inputenc}
\usepackage[T1]{fontenc}
\usepackage[ngerman,english]{babel}
%\usepackage{babelbib}

% mathematische Symbole
\usepackage{amsmath,amssymb,amsfonts,amstext}

% Listings
\usepackage{listings}
\lstset{numbers=left,numberstyle=\tiny,stepnumber=5,numbersep=5pt}

% erweiterte Zeichenbefehle
\usepackage{pst-all}

% Kopfzeilen frei gestaltbar
\usepackage{fancyhdr}
 

% Farben im Dokument m"oglich
\usepackage{color}

% Schriftart Helvetica
\usepackage{helvet}
\renewcommand{\familydefault}{phv}

% anderdhalbfacher Zeilenabstand
\usepackage{setspace}
\onehalfspacing

% Graphiken einbinden: hier f"ur pdflatex
\usepackage{graphicx}
\usepackage{import}
\usepackage{pdfpages}

% verbesserte Floating Plazierung
\usepackage{float}

% "Uberpr"ufung des Layouts
\usepackage{layout}
\usepackage{array}  % For better table formatting

% erweiterte Einstellungen der Bildunterschriften -> 8 Pt
\usepackage{caption}
\captionsetup{font=small,belowskip=12pt,aboveskip=4pt}
\usepackage{subcaption}

% f"ur floatbarrier
\usepackage{placeins}

\usepackage{ifthen}

% H"ohe und Breite des Textk"orpers etwas gr"osser definieren

% Quellcode
\sisetup{input-digits = 0123456789\pi}%

% Referenzierung
\usepackage{hyperref}

% kreise um zahlen/Buchstaben
\newcommand{\kreis}[1]{\unitlength1ex\begin{picture}(2.5,2.5)%
\put(0.75,0.75){\circle{2.5}}\put(0.75,0.75){\makebox(0,0){#1}}\end{picture}}

% f"ur inline figures
\usepackage{scalerel}

% Einr"uckung von und Abstand zwischen Abs"atzen
\setlength{\parindent}{0em}
\setlength{\parskip}{1.5ex plus0.5ex minus0.5ex}

% weniger Warnungen wegen "uberf"ullter Boxen
\tolerance = 9999
\sloppy

% Anpassung einiger "Uberschriften 
\renewcommand\figurename{Figure}
\renewcommand\tablename{Table}
%\newcommand{\unit}{\mathrm}

% Counter f"ur die Nummerierung
\newcounter{romancount}

% Boolsche Variable f"ur Bachelor-/Masterarbeit oder Bericht
\newboolean{thesis}

% custom command for parameters
\newcommand{\param}[1]{\textit{#1}}
